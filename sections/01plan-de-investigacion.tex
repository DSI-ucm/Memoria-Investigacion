\chapter{Plan de Investigación}

\section{Introducción}
% Descripción general del tipo de aplicación que quieres construir y de la problemática que intentan resolver
En la planificación de este proyecto se usará el modelo de proceso de \textbf{Diseño Guiado por Objetivos} descrito en clase para desarrollar una aplicación de gestión de despensa y facilitar la realización de listas de la compra. El ámbito de este documento es la planificación de la Fase de Investigación, en la que realizamos un Análisis de la Competencia y Entrevistas con usuarios potenciales.

\subsection{Definición del problema}
Se ha observado que uno de los mayores problemas en la conciliación de las tareas del hogar es la estructura centralizada de la asignación de tareas. Una de las personas de la casa se encarga de tener «en la cabeza» todo el inventario de la casa y las tareas por realizar, y las otras personas al no estar tan familiarizados con dichas tareas acaban cometiendo errores (comprar cebollas en lugar cebolletas, mayonesa de otra marca, etc.), por lo que suele ser más fácil para quien lleva la casa realizar las tareas directamente que delegarlas.

Tener una aplicación con la que manejar una de las partes de esas tareas con mayor carga mental (la de abastecimiento/inventariado) es fundamental si se desea realizar un reparto equitativo de las tareas del hogar.

\section{Selección de los usuarios}
Hemos identificado los siguientes tipos de usuarios:
\begin{itemize}
    \item \textbf{Persona que vive sola}: Le es fácil acordarse de todo lo que tiene y no necesita asignar compras a nadie, pero le viene bien usar una aplicación de este estilo para que no se le olvide nada al ir a hacer la compra.
    \item \textbf{Piso compartido}: La repartición de tareas es algo clave para este grupo de usuarios entre los que suele haber poca comunicación. Pueden notificar cuando se ha agotado algo para que otro compañero lo compre de camino a casa, o dividir los gastos al ver qué cosas ha comprado cada uno.
    \item \textbf{Familia}: Aunque la comunicación dentro de este grupo suele ser mejor, tener una aplicación enfocada a esto puede permitir saber en todo momento el consumo de determinados productos, e incluso incentivar el ahorro.
\end{itemize}

\section{Planificación de las entrevistas}

\begin{itemize}
    \item Hipótesis de personas y cómo se va a buscar a los usuarios
    \item Screener previo: unas pocas preguntas que sirven para saber en qué grupo de usuarios clasificar al entrevistado y, por tanto, qué preguntas hacer a continuación
    \item Introducción a la entrevista
    \item Consentimiento para grabar
    \item Guiones de las entrevistas (posiblemente varios modelos)
\end{itemize}

\section{Otros tipos de técnicas utilizadas para la elaboración del estudio}
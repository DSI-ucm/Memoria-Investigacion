\chapter{Transcripción de las entrevistas}
\label{ch:entrevistas}
\section{Entrevista 1}
\begin{description}
    \preg ¿Cuál es su edad?
    \preg ¿Convive con más gente en su hogar? ¿Cuál es su relación con ellos?
    \preg ¿Tiene un smartphone o Tablet? ¿Es Android o iOS?
    \preg ¿Utiliza algún medio para anotar los productos que necesite comprar? ¿Cual?
    \preg Si no utiliza ningún medio, ¿Suele olvidarse de artículos que necesita comprar?
    \preg ¿Su móvil es de gama baja, media o alta?
    \preg ¿Con que frecuencia haces la compra? (diaria, semanal, mensual)
    \preg ¿Sueles comprar por internet? (Amazon Now, Carrefour Online, etc)
    \preg Si sueles comprar por internet, ¿Vas a recogerlo a tienda o pides envío?
    \preg Si vas a comprar al sitio, ¿Lo traes inmediatamente de vuelta (coche, en mano, transporte publico), o pides que te lo envíen?
    \preg ¿Pierdes mucho tiempo yendo a comprar, cuanto tiempo?
    \preg ¿Qué es lo que te hace perder el tiempo? ¿Buscar las cosas? ¿El transporte al sitio? ¿la caja?
    \preg ¿Sabes que es lo que tienes que comprar para la casa, o que es lo que hace falta? ¿Como lo recuerdas?
    \preg ¿Sabes que tipos de productos suele haber en tu casa con regularidad?
    \preg ¿Sabes en que lugares tienen los productos que necesitas?
    \preg ¿Sueles comprar mas económico, o por calidad del producto?
    \preg ¿Cuando vas a comprar, si no sabes con exactitud que producto escoger preguntas a alguien? (otros residentes de casa, trabajadores del super)
    \preg ¿Que partes del proceso de hacer la compra odias mas? ¿Cuales prefieres hacer? ¿Por qué? (comprar, organizar la compra, realizar el listado, etc.)
    \preg ¿Cuando vas a comprar usas alguna app para buscar información del producto que compras? (valores nutricionales, precio, valoraciones, etc). Si es así, ¿cuales utilizas?
    \preg ¿Sueles comprar cosas innecesarias?
    \preg ¿Suele tener mas de una lista de la compra? (Por ej. Lista de alimentos, lista de drogueria, etc) 
    \preg ¿Suele tener listas de la compra compartidas con diferentes grupos de personas? (por ej. lista de casa, lista para una fiesta, lista para cosas del trabajo, etc)
\end{description}    

\section{Entrevista 2}
\begin{description}
    \preg ¿Cuál es su edad?
    \preg ¿Convive con más gente en su hogar? ¿Cuál es su relación con ellos?
    \preg ¿Tiene un smartphone o Tablet? ¿Es Android o iOS?
    \preg ¿Utiliza algún medio para anotar los productos que necesite comprar? ¿Cual?
    \preg Si no utiliza ningún medio, ¿Suele olvidarse de artículos que necesita comprar?
    \preg ¿Su móvil es de gama baja, media o alta?
    \preg ¿Con que frecuencia haces la compra? (diaria, semanal, mensual)
    \preg ¿Sueles comprar por internet? (Amazon Now, Carrefour Online, etc)
    \preg Si sueles comprar por internet, ¿Vas a recogerlo a tienda o pides envío?
    \preg Si vas a comprar al sitio, ¿Lo traes inmediatamente de vuelta (coche, en mano, transporte publico), o pides que te lo envíen?
    \preg ¿Pierdes mucho tiempo yendo a comprar, cuanto tiempo?
    \preg ¿Qué es lo que te hace perder el tiempo? ¿Buscar las cosas? ¿El transporte al sitio? ¿la caja?
    \preg ¿Sabes que es lo que tienes que comprar para la casa, o que es lo que hace falta? ¿Como lo recuerdas?
    \preg ¿Sabes que tipos de productos suele haber en tu casa con regularidad?
    \preg ¿Sabes en que lugares tienen los productos que necesitas?
    \preg ¿Sueles comprar mas económico, o por calidad del producto?
    \preg ¿Cuando vas a comprar, si no sabes con exactitud que producto escoger preguntas a alguien? (otros residentes de casa, trabajadores del super)
    \preg ¿Que partes del proceso de hacer la compra odias mas? ¿Cuales prefieres hacer? ¿Por qué? (comprar, organizar la compra, realizar el listado, etc.)
    \preg ¿Cuando vas a comprar usas alguna app para buscar información del producto que compras? (valores nutricionales, precio, valoraciones, etc). Si es así, ¿cuales utilizas?
    \preg ¿Sueles comprar cosas innecesarias?
    \preg ¿Suele tener mas de una lista de la compra? (Por ej. Lista de alimentos, lista de drogueria, etc) 
    \preg ¿Suele tener listas de la compra compartidas con diferentes grupos de personas? (por ej. lista de casa, lista para una fiesta, lista para cosas del trabajo, etc)
\end{description}

\section{Entrevista 3 - Paloma}
\setresp{Paloma}
\begin{description}
    \preg ¿Cuál es su edad?
    \resp 24 años
    \preg ¿Convive con más gente en su hogar? ¿Cuál es su relación con ellos?
    \resp Vivo sola
    \preg ¿Tiene un smartphone o Tablet? ¿Es Android o iOS?
    \resp Tengo un smartphone Android.
    \preg ¿Utiliza algún medio para anotar los productos que necesite comprar? ¿Cual?
    \resp Utilizo el bloc de notas del móvil.
    \preg Si no utiliza ningún medio, ¿Suele olvidarse de artículos que necesita comprar?
    \resp Aun así se me suele olvidar muchos productos porque no los anoto en el bloc de notas.
    \preg ¿Su móvil es de gama baja, media o alta?
    \resp Gama media
    \preg ¿Con que frecuencia haces la compra? (diaria, semanal, mensual)
    \resp 2 veces por semana.
    \preg ¿Sueles comprar por internet? (Amazon Now, Carrefour Online, etc)
    \resp Nunca compro por internet, ya que prefiero escoger los alimentos que voy a consumir yo misma.
    \preg Si sueles comprar por internet, ¿Vas a recogerlo a tienda o pides envío?
    \resp 
    \preg Si vas a comprar al sitio, ¿Lo traes inmediatamente de vuelta (coche, en mano, transporte publico), o pides que te lo envíen?
    \resp Siempre inmediatamente y con la mano.
    \preg ¿Pierdes mucho tiempo yendo a comprar, cuanto tiempo?
    \resp Unos 30 minutos, vivo cerca del supermercado
    \preg ¿Qué es lo que te hace perder el tiempo? ¿Buscar las cosas? ¿El transporte al sitio? ¿la caja?
    \resp Sobre todo la caja, y en mirar productos que no tenía anotados.
    \preg ¿Sabes que es lo que tienes que comprar para la casa, o que es lo que hace falta? ¿Como lo recuerdas?
    \resp Intento siempre anotarlo todo y estar pendiente, aunque se me olvidan bastantes cosas.
    \preg ¿Sabes que tipos de productos suele haber en tu casa con regularidad?
    \resp Siempre suelo saberlo, soy bastante atenta para eso.
    \preg ¿Sabes en que lugares tienen los productos que necesitas?
    \resp Si, salvo que sea algo nuevo.
    \preg ¿Sueles comprar mas económico, o por calidad del producto?
    \resp Casi siempre busco la calidad del producto por encima del precio.
    \preg ¿Cuando vas a comprar, si no sabes con exactitud que producto escoger preguntas a alguien? (otros residentes de casa, trabajadores del super)
    \resp Pregunto a familiares que sé que entienden de esos productos.
    \preg ¿Que partes del proceso de hacer la compra odias mas? ¿Cuales prefieres hacer? ¿Por qué? (comprar, organizar la compra, realizar el listado, etc.)
    \resp Lo que mas odio es el pensar que tengo que ir, sobre todo en invierno cuando hace más frio. Lo que prefiero es el ver la nevera llena y saber que no tengo que volver hasta dentro de un par de días.
    Organizarlos no me preocupa demasiado.
    \preg ¿Cuando vas a comprar usas alguna app para buscar información del producto que compras? (valores nutricionales, precio, valoraciones, etc). Si es así, ¿cuales utilizas?
    \resp No utilizo ninguna app, ni nada por el estilo.
    \preg ¿Sueles comprar cosas innecesarias?
    \resp No suelo, aunque alguna vez lo he hecho.
    \preg ¿Suele tener mas de una lista de la compra? (Por ej. Lista de alimentos, lista de drogueria, etc) 
    \resp No.
    \preg ¿Suele tener listas de la compra compartidas con diferentes grupos de personas? (por ej. lista de casa, lista para una fiesta, lista para cosas del trabajo, etc)
    \resp Si es para alguna fiesta si, aunque con el covid ya no.
\end{description}

\section{Entrevista 4 - Emilio Sevilla}
\setresp{Emilio}
\begin{description}
    \preg ¿Cuál es su edad?
    \resp  Tengo 24 años
    \preg ¿Convive con más gente en su hogar? ¿Cuál es su relación con ellos?
    \resp  No, vivo sólo. 
    \preg ¿Tiene un smartphone o Tablet? ¿Es Android o iOS?
    \resp  Tengo un iphone.
    \preg ¿Utiliza algún medio para anotar los productos que necesite comprar? ¿Cual?
    \resp  Suelo apuntarlo en las notas del móvil
    \preg Si no utiliza ningún medio, ¿Suele olvidarse de artículos que necesita comprar?
    \resp  Pues como ya te he dicho si que utilizo un medio pero aun así se me suelen olvidar cosas.
    \preg ¿Su móvil es de gama baja, media o alta?
    \resp  Es de gama alta.
    \preg ¿Con que frecuencia haces la compra? (diaria, semanal, mensual)
    \resp  Depende pero por lo general semanalmente aunque como se me suele olvidar alguna cosa pues hay varios días que tengo que bajar a comprar.
    \preg ¿Sueles comprar por internet? (Amazon Now, Carrefour Online, etc)
    \resp  La compra nunca la hago por internet, pero para ropa por ejemplo si que lo hago, es decir, no me asustaría hacerlo.
    \preg Si sueles comprar por internet, ¿Vas a recogerlo a tienda o pides envío?
    \resp  Pues ya que lo compro por internet que me lo traigan a casa ¿no? si no me saltaría ese paso.
    \preg Si vas a comprar al sitio, ¿Lo traes inmediatamente de vuelta (coche, en mano, transporte publico), o pides que te lo envíen?
    \resp  Me lo llevo aunque pese mucho, si tengo que parar a descansar lo hago, eso nunca ha sido un problema para mi.
    \preg ¿Pierdes mucho tiempo yendo a comprar, cuanto tiempo?
    \resp  Pues un poco si, nunca organizo la lista de la compra en función a como están colocados los artículos en el super.
    \preg ¿Qué es lo que te hace perder el tiempo? ¿Buscar las cosas? ¿El transporte al sitio? ¿la caja?
    \resp  Como ya te he dicho pierdo tiempo en buscar las cosas. Si pudiera organizarlo por pasillos o algo así seguro que tardaría menos.
    \preg ¿Sabes que es lo que tienes que comprar para la casa, o que es lo que hace falta? ¿Como lo recuerdas?
    \resp  Pues por lo general durante la semana me voy dando cuenta de que necesito cosas e intento apuntarlo, pero siempre antes de ir a comprar abro la nevera, la despensa e intento pensar en qué me hace falta.
    \preg ¿Sabes que tipos de productos suele haber en tu casa con regularidad?
    \resp  Sí claro, los compro yo todos.
    \preg ¿Sabes en que lugares tienen los productos que necesitas?
    \resp  Si a no ser que sea algo nuevo o una marca específica que me han recomendado y no sé dónde la puedan vender cerca de mi casa.
    \preg ¿Sueles comprar mas económico, o por calidad del producto?
    \resp  Pues depende algunas cosas que uso más me interesa la calidad y otras que uso menos que sean baratas.
    \preg ¿Cuando vas a comprar, si no sabes con exactitud que producto escoger preguntas a alguien? (otros residentes de casa, trabajadores del super)
    \resp  Pues suelo preguntar al dependiente que productos son mejores y por qué. Pero a veces me gusta comprar uno, el próximo día que voy pruebo el otro y así decido cual comprar habitualmente.
    \preg ¿Que partes del proceso de hacer la compra odias mas? ¿Cuales prefieres hacer? ¿Por qué? (comprar, organizar la compra, realizar el listado, etc.)
    \resp  Pues claramente hacer cola y luego cargar hasta mi casa con ello. Prefiero la parte en la que me doy paseos por el super (risas).
    \preg ¿Cuando vas a comprar usas alguna app para buscar información del producto que compras? (valores nutricionales, precio, valoraciones, etc). Si es así, ¿cuales utilizas?
    \resp  Una vez me dijeron una app que escaneas el código de barras y te dice si el producto es bueno y eso. Pero nunca me la llegué a descargar.
    \preg ¿Sueles comprar cosas innecesarias?
    \resp  Alguna vez pero por lo general no.
    \preg ¿Suele tener mas de una lista de la compra? (Por ej. Lista de alimentos, lista de drogueria, etc) 
    \resp  Que va, todo junto de ahí los problemas (risas)
    \preg ¿Suele tener listas de la compra compartidas con diferentes grupos de personas? (por ej. lista de casa, lista para una fiesta, lista para cosas del trabajo, etc)
    \resp  Sí cuando nos vamos de barbacoa los amigos la hacemos por el grupo de whatsapp, pero siempre se olvidan cosas. Molaría tener una guardada por defecto para que no pasase eso.
\end{description}